\documentclass[a4paper,11pt]{article}

%\usepackage[utf8]{inputenc}
%\usepackage[T1]{fontenc}
%\usepackage{xunicode}
%\usepackage{listings}
\usepackage[french]{babel}
%\usepackage{url}
%\usepackage{times}
\usepackage{fontspec}
\defaultfontfeatures{Mapping=tex-text,Scale=MatchLowercase}
\usepackage{a4wide}
\usepackage{verbatim}
\usepackage{polyglossia}
\setdefaultlanguage{french}
\usepackage{minted}
\usepackage{graphicx}
%\usepackage{graphviz}
\usepackage{amssymb}
\usepackage{float}

\usepackage{geometry}
\geometry{hmargin=2.5cm,vmargin=1.5cm}

\graphicspath{{img/}}

\title{\includegraphics{inp-enseeiht.png} \\ Tribunal de Grande Instance de l’N7 \\ ~ \\ Rapport d’expertise}
\author{Élie {\sc Bouttier}\\Yvan {\sc Tchomgue}}
\date\today

\begin{document}

\maketitle

\vspace{2cm}

\begin{abstract}
    N° du Parquet : 1015800108

    N° Instruction : 3/11/31

    Procédure correctionnelle

    Personne mise en examun : M. {\sc Zouzou}

    Qualifications : Vol de voiture

    Mission : expertise d’une clef USB de marque « Scandisk »
\end{abstract}

\vspace{2cm}

\tableofcontents

\newpage

\section{Outils et commandes nécessaires}

\begin{itemize}
    \item \texttt{exiftool} pour l'analyse des images
    \item \texttt{ls -l} pour vérifier les dernières dates de modifications d'un
        fichier
    \item \texttt{file} pour connaître le type réel d'un fichier
    \item \texttt{xxd} convertit en représentation hexadécimale et inversement
    \item \texttt{photorec} pour récupérer les fichiers supprimés d'un disque
        mais on perd le nom du fichier
    \item \texttt{sleuthkit} boîte à outils inforensique
\end{itemize}

\end{document}
